%% Generated by Sphinx.
\def\sphinxdocclass{report}
\documentclass[letterpaper,10pt,spanish]{sphinxmanual}
\ifdefined\pdfpxdimen
   \let\sphinxpxdimen\pdfpxdimen\else\newdimen\sphinxpxdimen
\fi \sphinxpxdimen=.75bp\relax
\ifdefined\pdfimageresolution
    \pdfimageresolution= \numexpr \dimexpr1in\relax/\sphinxpxdimen\relax
\fi
%% let collapsible pdf bookmarks panel have high depth per default
\PassOptionsToPackage{bookmarksdepth=5}{hyperref}

\PassOptionsToPackage{warn}{textcomp}
\usepackage[utf8]{inputenc}
\ifdefined\DeclareUnicodeCharacter
% support both utf8 and utf8x syntaxes
  \ifdefined\DeclareUnicodeCharacterAsOptional
    \def\sphinxDUC#1{\DeclareUnicodeCharacter{"#1}}
  \else
    \let\sphinxDUC\DeclareUnicodeCharacter
  \fi
  \sphinxDUC{00A0}{\nobreakspace}
  \sphinxDUC{2500}{\sphinxunichar{2500}}
  \sphinxDUC{2502}{\sphinxunichar{2502}}
  \sphinxDUC{2514}{\sphinxunichar{2514}}
  \sphinxDUC{251C}{\sphinxunichar{251C}}
  \sphinxDUC{2572}{\textbackslash}
\fi
\usepackage{cmap}
\usepackage[T1]{fontenc}
\usepackage{amsmath,amssymb,amstext}
\usepackage{babel}



\usepackage{tgtermes}
\usepackage{tgheros}
\renewcommand{\ttdefault}{txtt}



\usepackage[Sonny]{fncychap}
\ChNameVar{\Large\normalfont\sffamily}
\ChTitleVar{\Large\normalfont\sffamily}
\usepackage{sphinx}

\fvset{fontsize=auto}
\usepackage{geometry}


% Include hyperref last.
\usepackage{hyperref}
% Fix anchor placement for figures with captions.
\usepackage{hypcap}% it must be loaded after hyperref.
% Set up styles of URL: it should be placed after hyperref.
\urlstyle{same}

\addto\captionsspanish{\renewcommand{\contentsname}{Contents:}}

\usepackage{sphinxmessages}
\setcounter{tocdepth}{1}



\title{Konrad Gourmet}
\date{23 de octubre de 2021}
\release{0.0}
\author{Jonatan Ahumada}
\newcommand{\sphinxlogo}{\vbox{}}
\renewcommand{\releasename}{Versión}
\makeindex
\begin{document}

\ifdefined\shorthandoff
  \ifnum\catcode`\=\string=\active\shorthandoff{=}\fi
  \ifnum\catcode`\"=\active\shorthandoff{"}\fi
\fi

\pagestyle{empty}
\sphinxmaketitle
\pagestyle{plain}
\sphinxtableofcontents
\pagestyle{normal}
\phantomsection\label{\detokenize{index::doc}}



\chapter{SRS}
\label{\detokenize{SRS:srs}}\label{\detokenize{SRS::doc}}

\section{Introducción}
\label{\detokenize{SRS:introduccion}}
\sphinxAtStartPar
El propósito de este SRS es
describir los requerimientos
funcionales y no funcionales
del sistema de abastecimiento
de alimentos para la cadena
Konrad Gourmet. Este documento
está dirigido a los miembros
del equipo de desarrollo
y a los interesados. Los
documentos especificados
aquí serán incluidos en el release
1.0


\section{Ámbito del proyecto}
\label{\detokenize{SRS:ambito-del-proyecto}}
\sphinxAtStartPar
Konrad Gorumet es una cadena de
restaurantes que cuenta con varias sucurales.
Anteriormente, el proceso para consultar y
actualizar su abastecimiento era manual,
por lo que era imposible consultar
cifras exactas en tiempo real de todo
su inventario (los alimentos necesarios para
los platos que ofrecen a su cliente final). Esto ocasionaba pérdidas
por contar con cifras extactas con las
que solicitar cotizaciones a sus proveedores.

\sphinxAtStartPar
A raiz de eso, surgió la necesidad de implementar
un sistema que permita seguir el inventario
de las sucursales de Konrad Gorumet y que también automatize
y optimize las negociociones con sus proveedores.


\section{Descripción general}
\label{\detokenize{SRS:descripcion-general}}
\sphinxAtStartPar
Konrad Gourmet es un sistema para
cuantificar el inventario de
varias sucursales de comida.
El sistema genera órdenes de compras,
facturas, y resúmenes del inventario
de distintas sucursales de Konrad
Gourmet. Una vez generada una cotización,
el sistema le provee al Director de Compras
facilidades para validar si la respuesta del
proveedor es factible o no, de acuerdo
a la respuesta de otros proveedores, así
como a los precios de los alimentos
publicados por la entidad gubernamental
correspondiente.


\section{Funcionalidades del sistema}
\label{\detokenize{SRS:funcionalidades-del-sistema}}
\sphinxAtStartPar
Los siguientes casos de uso
fueron identificados. Se clasifican
se agruparon para facilitar
su comprensión en 4 áreas:
\begin{enumerate}
\sphinxsetlistlabels{\arabic}{enumi}{enumii}{}{.}%
\item {} 
\sphinxAtStartPar
Sucursal

\item {} 
\sphinxAtStartPar
Central

\item {} 
\sphinxAtStartPar
Opciones de configuración

\item {} 
\sphinxAtStartPar
Eventos automáticos o de soporte

\end{enumerate}


\subsection{Sucursal}
\label{\detokenize{SRS:sucursal}}
\noindent\sphinxincludegraphics[width=500\sphinxpxdimen]{{casos_de_uso_001}.png}


\subsection{Central}
\label{\detokenize{SRS:central}}
\noindent\sphinxincludegraphics[width=500\sphinxpxdimen]{{casos_de_uso_002}.png}

\noindent\sphinxincludegraphics[width=500\sphinxpxdimen]{{casos_de_uso_003}.png}


\subsection{Opciones de configuración}
\label{\detokenize{SRS:opciones-de-configuracion}}
\noindent\sphinxincludegraphics[width=500\sphinxpxdimen]{{casos_de_uso_004}.png}


\subsection{Eventos automáticos o procesos de soporte}
\label{\detokenize{SRS:eventos-automaticos-o-procesos-de-soporte}}
\noindent\sphinxincludegraphics[width=500\sphinxpxdimen]{{casos_de_uso}.png}


\section{Requerimientos de interfaces externas}
\label{\detokenize{SRS:requerimientos-de-interfaces-externas}}

\subsection{Interfaces de software}
\label{\detokenize{SRS:interfaces-de-software}}\begin{itemize}
\item {} 
\sphinxAtStartPar
La tecnología utilizada debe ser de libre licenciamiento

\item {} 
\sphinxAtStartPar
Los lenguajes, frameworks y librerías deben ser las últimas versiones estables, reconocidas, con\sphinxhyphen{} soporte y de buenas prácticas

\item {} 
\sphinxAtStartPar
Cualquier servicio que se tenga que exponer hacia un sistema externo, se debe hacer a través de SOAP

\end{itemize}


\section{Atributos de calidad}
\label{\detokenize{SRS:atributos-de-calidad}}

\subsection{Requerimientos de usabilidad}
\label{\detokenize{SRS:requerimientos-de-usabilidad}}\begin{itemize}
\item {} 
\sphinxAtStartPar
El sistema debe ser “responisive”, para ser utilizado desde dispositivos móviles

\item {} 
\sphinxAtStartPar
El sistema debe poder cambiar de imagen corporativa de manera parametrizada, sin necesidad de

\end{itemize}

\sphinxAtStartPar
recurrir a un diseñador gráfico


\subsection{Requerimientos de confiabilidad}
\label{\detokenize{SRS:requerimientos-de-confiabilidad}}\begin{itemize}
\item {} 
\sphinxAtStartPar
El sistema debe poder recuperarse de desastres, debe tener un centro de datos alterno

\end{itemize}


\subsection{Requerimientos de disponibilidad}
\label{\detokenize{SRS:requerimientos-de-disponibilidad}}\begin{itemize}
\item {} 
\sphinxAtStartPar
El sistema debe ofrecer un 99,7\% de alta disponibilidad

\end{itemize}


\subsection{Requerimientos de desempeño}
\label{\detokenize{SRS:requerimientos-de-desempeno}}\begin{itemize}
\item {} 
\sphinxAtStartPar
El sistema debe estar en capacidad de atender a 200.000 usuarios concurrentes

\item {} 
\sphinxAtStartPar
El sisteba debe estar en capacidad de  procesar 1000 TPS

\end{itemize}


\subsection{Requerimientos de seguridad}
\label{\detokenize{SRS:requerimientos-de-seguridad}}\begin{itemize}
\item {} 
\sphinxAtStartPar
El sistema debe contar con un módulo de autenticación y autorización

\item {} 
\sphinxAtStartPar
El sistema debe poder manejar usuarios, perfiles o roles y permisos

\item {} 
\sphinxAtStartPar
Cualquier comunicación debe estar asegurada con protocolo HTTPS

\item {} 
\sphinxAtStartPar
La contraseña debe cumplir con un patrón de mínimo 8 caracteres que incluyan una mayúscula, una minúscula y un número

\item {} 
\sphinxAtStartPar
La contraseña debe ser almacenada encripatada con algún algoritmo estándar

\item {} 
\sphinxAtStartPar
Los correos emitidos deben estar certificados y con estampa cronológica

\end{itemize}


\subsection{Requerimientos de mantenimiento}
\label{\detokenize{SRS:requerimientos-de-mantenimiento}}\begin{itemize}
\item {} 
\sphinxAtStartPar
Se espera un crecimiento de 200\% en el almacenamiento de los documentos, archivos de carga

\end{itemize}

\sphinxAtStartPar
y la data
\sphinxhyphen{} Se debe realizar backup diario de la base de datos
\sphinxhyphen{} Se debe contemplar el cambio fácil de motor de base de datos
\sphinxhyphen{} Cada acción del CRUD debe registrar su respectivo movimiento de auditoría con la
siguiente información: 1) Acción, 2) Usuario, 3) Fecha, 4) Hora
\sphinxhyphen{} Cada error producido en el sistema debe quedar registrado en un log


\subsection{Requerimientos de portabilidad}
\label{\detokenize{SRS:requerimientos-de-portabilidad}}
\sphinxAtStartPar
No se solicitaron requerimientos de portabilidad.


\chapter{Requerimientos específicos}
\label{\detokenize{SRS:requerimientos-especificos}}

\begin{savenotes}\sphinxattablestart
\centering
\begin{tabular}[t]{|\X{25}{125}|\X{100}{125}|}
\hline
\sphinxstyletheadfamily 
\sphinxAtStartPar
UC:
&\sphinxstyletheadfamily 
\sphinxAtStartPar
Registrar platos
\\
\hline
\sphinxAtStartPar
Actores:
&
\sphinxAtStartPar
Jefe de Cocina
\\
\hline
\sphinxAtStartPar
Procesamiento:
&
\sphinxAtStartPar
El jefe de cocina de cada uno de los puntos del restaurante en la ciudad puede registrar los platos o menú de este para lo cual debe ingresar los siguientes datos
\\
\hline
\sphinxAtStartPar
Salidas:
&\\
\hline
\sphinxAtStartPar
Excepciones:
&\\
\hline
\end{tabular}
\par
\sphinxattableend\end{savenotes}


\begin{savenotes}\sphinxattablestart
\centering
\sphinxcapstartof{table}
\sphinxthecaptionisattop
\sphinxcaption{Prueba}\label{\detokenize{SRS:id1}}
\sphinxaftertopcaption
\begin{tabular}[t]{|\X{20}{100}|\X{20}{100}|\X{20}{100}|\X{20}{100}|\X{20}{100}|}
\hline
\sphinxstyletheadfamily 
\sphinxAtStartPar
UC
&\sphinxstyletheadfamily 
\sphinxAtStartPar
Actores
&\sphinxstyletheadfamily 
\sphinxAtStartPar
Procesamiento
&\sphinxstyletheadfamily 
\sphinxAtStartPar
Salidas
&\sphinxstyletheadfamily 
\sphinxAtStartPar
Excepciones
\\
\hline
\sphinxAtStartPar
Registrar Platos
&
\sphinxAtStartPar
Jefe de cocina
&
\sphinxAtStartPar
El jefe de cocina de cada uno de los puntos del restaurante en la ciudad puede registrar los platos o menu de este para lo cual debe ingresar los siguientes datos
&
\sphinxAtStartPar
El plato se registra y se descuentan undades de producto
&
\sphinxAtStartPar
La vista se recarga nuevamente, pero se muestran errores de validación por cada campo del formulario
\\
\hline
\end{tabular}
\par
\sphinxattableend\end{savenotes}


\chapter{Indices and tables}
\label{\detokenize{index:indices-and-tables}}\begin{itemize}
\item {} 
\sphinxAtStartPar
\DUrole{xref,std,std-ref}{genindex}

\item {} 
\sphinxAtStartPar
\DUrole{xref,std,std-ref}{modindex}

\item {} 
\sphinxAtStartPar
\DUrole{xref,std,std-ref}{search}

\end{itemize}



\renewcommand{\indexname}{Índice}
\printindex
\end{document}