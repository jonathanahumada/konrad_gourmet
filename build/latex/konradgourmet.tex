%% Generated by Sphinx.
\def\sphinxdocclass{report}
\documentclass[letterpaper,10pt,spanish]{sphinxmanual}
\ifdefined\pdfpxdimen
   \let\sphinxpxdimen\pdfpxdimen\else\newdimen\sphinxpxdimen
\fi \sphinxpxdimen=.75bp\relax
\ifdefined\pdfimageresolution
    \pdfimageresolution= \numexpr \dimexpr1in\relax/\sphinxpxdimen\relax
\fi
%% let collapsible pdf bookmarks panel have high depth per default
\PassOptionsToPackage{bookmarksdepth=5}{hyperref}

\PassOptionsToPackage{warn}{textcomp}
\usepackage[utf8]{inputenc}
\ifdefined\DeclareUnicodeCharacter
% support both utf8 and utf8x syntaxes
  \ifdefined\DeclareUnicodeCharacterAsOptional
    \def\sphinxDUC#1{\DeclareUnicodeCharacter{"#1}}
  \else
    \let\sphinxDUC\DeclareUnicodeCharacter
  \fi
  \sphinxDUC{00A0}{\nobreakspace}
  \sphinxDUC{2500}{\sphinxunichar{2500}}
  \sphinxDUC{2502}{\sphinxunichar{2502}}
  \sphinxDUC{2514}{\sphinxunichar{2514}}
  \sphinxDUC{251C}{\sphinxunichar{251C}}
  \sphinxDUC{2572}{\textbackslash}
\fi
\usepackage{cmap}
\usepackage[T1]{fontenc}
\usepackage{amsmath,amssymb,amstext}
\usepackage{babel}



\usepackage{tgtermes}
\usepackage{tgheros}
\renewcommand{\ttdefault}{txtt}



\usepackage[Sonny]{fncychap}
\ChNameVar{\Large\normalfont\sffamily}
\ChTitleVar{\Large\normalfont\sffamily}
\usepackage{sphinx}

\fvset{fontsize=auto}
\usepackage{geometry}


% Include hyperref last.
\usepackage{hyperref}
% Fix anchor placement for figures with captions.
\usepackage{hypcap}% it must be loaded after hyperref.
% Set up styles of URL: it should be placed after hyperref.
\urlstyle{same}

\addto\captionsspanish{\renewcommand{\contentsname}{Contents:}}

\usepackage{sphinxmessages}
\setcounter{tocdepth}{1}



\title{Konrad Gourmet}
\date{09 de noviembre de 2021}
\release{0.0}
\author{Jonatan Ahumada, Jorge Garzón}
\newcommand{\sphinxlogo}{\vbox{}}
\renewcommand{\releasename}{Versión}
\makeindex
\begin{document}

\ifdefined\shorthandoff
  \ifnum\catcode`\=\string=\active\shorthandoff{=}\fi
  \ifnum\catcode`\"=\active\shorthandoff{"}\fi
\fi

\pagestyle{empty}
\sphinxmaketitle
\pagestyle{plain}
\sphinxtableofcontents
\pagestyle{normal}
\phantomsection\label{\detokenize{index::doc}}


\begin{figure}[htbp]
\centering
\capstart

\noindent\sphinxincludegraphics[width=600\sphinxpxdimen]{{cronograma}.png}
\caption{Cronograma del proyecto}\label{\detokenize{index:id1}}
\begin{sphinxlegend}\begin{quote}
\end{quote}
\end{sphinxlegend}
\end{figure}


\chapter{SRS}
\label{\detokenize{SRS:srs}}\label{\detokenize{SRS::doc}}

\section{Introducción}
\label{\detokenize{SRS:introduccion}}
\sphinxAtStartPar
El propósito de este SRS es
describir los requerimientos
funcionales y no funcionales
del sistema de abastecimiento
de alimentos para la cadena
Konrad Gourmet. Este documento
está dirigido a los miembros
del equipo de desarrollo
y a los interesados. Los
documentos especificados
aquí serán incluidos en el release
1.0


\section{Ámbito del proyecto}
\label{\detokenize{SRS:ambito-del-proyecto}}
\sphinxAtStartPar
Konrad Gorumet es una cadena de
restaurantes que cuenta con varias sucurales.
Anteriormente, el proceso para consultar y
actualizar su abastecimiento era manual,
por lo que era imposible consultar
cifras exactas en tiempo real de todo
su inventario (los alimentos necesarios para
los platos que ofrecen a su cliente final). Esto ocasionaba pérdidas
por contar con cifras extactas con las
que solicitar cotizaciones a sus proveedores.

\sphinxAtStartPar
A raiz de eso, surgió la necesidad de implementar
un sistema que permita seguir el inventario
de las sucursales de Konrad Gorumet y que también automatize
y optimize las negociociones con sus proveedores.


\section{Descripción general}
\label{\detokenize{SRS:descripcion-general}}
\sphinxAtStartPar
Konrad Gourmet es un sistema para
cuantificar el inventario de
varias sucursales de comida.
El sistema genera órdenes de compras,
facturas, y resúmenes del inventario
de distintas sucursales de Konrad
Gourmet. Una vez generada una cotización,
el sistema le provee al Director de Compras
facilidades para validar si la respuesta del
proveedor es factible o no, de acuerdo
a la respuesta de otros proveedores, así
como a los precios de los alimentos
publicados por la entidad gubernamental
correspondiente.


\section{Funcionalidades del sistema}
\label{\detokenize{SRS:funcionalidades-del-sistema}}
\sphinxAtStartPar
Los siguientes casos de uso
fueron identificados. Se clasifican
se agruparon para facilitar
su comprensión en 4 áreas:
\begin{enumerate}
\sphinxsetlistlabels{\arabic}{enumi}{enumii}{}{.}%
\item {} 
\sphinxAtStartPar
Sucursal

\item {} 
\sphinxAtStartPar
Central

\item {} 
\sphinxAtStartPar
Opciones de configuración

\item {} 
\sphinxAtStartPar
Eventos automáticos o de soporte

\end{enumerate}


\subsection{Sucursal}
\label{\detokenize{SRS:sucursal}}
\noindent\sphinxincludegraphics[width=500\sphinxpxdimen]{{casos_de_uso_001}.png}


\subsection{Central}
\label{\detokenize{SRS:central}}
\noindent\sphinxincludegraphics[width=500\sphinxpxdimen]{{casos_de_uso_002}.png}

\noindent\sphinxincludegraphics[width=500\sphinxpxdimen]{{casos_de_uso_003}.png}


\subsection{Opciones de configuración}
\label{\detokenize{SRS:opciones-de-configuracion}}
\noindent\sphinxincludegraphics[width=500\sphinxpxdimen]{{casos_de_uso_004}.png}


\subsection{Eventos automáticos o procesos de soporte}
\label{\detokenize{SRS:eventos-automaticos-o-procesos-de-soporte}}
\noindent\sphinxincludegraphics[width=500\sphinxpxdimen]{{casos_de_uso}.png}


\section{Requerimientos de interfaces externas}
\label{\detokenize{SRS:requerimientos-de-interfaces-externas}}

\subsection{Interfaces de software}
\label{\detokenize{SRS:interfaces-de-software}}\begin{itemize}
\item {} 
\sphinxAtStartPar
La tecnología utilizada debe ser de libre licenciamiento

\item {} 
\sphinxAtStartPar
Los lenguajes, frameworks y librerías deben ser las últimas versiones estables, reconocidas, con\sphinxhyphen{} soporte y de buenas prácticas

\item {} 
\sphinxAtStartPar
Cualquier servicio que se tenga que exponer hacia un sistema externo, se debe hacer a través de SOAP

\end{itemize}


\section{Atributos de calidad}
\label{\detokenize{SRS:atributos-de-calidad}}

\subsection{Requerimientos de usabilidad}
\label{\detokenize{SRS:requerimientos-de-usabilidad}}\begin{itemize}
\item {} 
\sphinxAtStartPar
El sistema debe ser “responisive”, para ser utilizado desde dispositivos móviles

\item {} 
\sphinxAtStartPar
El sistema debe poder cambiar de imagen corporativa de manera parametrizada, sin necesidad de

\end{itemize}

\sphinxAtStartPar
recurrir a un diseñador gráfico


\subsection{Requerimientos de confiabilidad}
\label{\detokenize{SRS:requerimientos-de-confiabilidad}}\begin{itemize}
\item {} 
\sphinxAtStartPar
El sistema debe poder recuperarse de desastres, debe tener un centro de datos alterno

\end{itemize}


\subsection{Requerimientos de disponibilidad}
\label{\detokenize{SRS:requerimientos-de-disponibilidad}}\begin{itemize}
\item {} 
\sphinxAtStartPar
El sistema debe ofrecer un 99,7\% de alta disponibilidad

\end{itemize}


\subsection{Requerimientos de desempeño}
\label{\detokenize{SRS:requerimientos-de-desempeno}}\begin{itemize}
\item {} 
\sphinxAtStartPar
El sistema debe estar en capacidad de atender a 200.000 usuarios concurrentes

\item {} 
\sphinxAtStartPar
El sisteba debe estar en capacidad de  procesar 1000 TPS

\end{itemize}


\subsection{Requerimientos de seguridad}
\label{\detokenize{SRS:requerimientos-de-seguridad}}\begin{itemize}
\item {} 
\sphinxAtStartPar
El sistema debe contar con un módulo de autenticación y autorización

\item {} 
\sphinxAtStartPar
El sistema debe poder manejar usuarios, perfiles o roles y permisos

\item {} 
\sphinxAtStartPar
Cualquier comunicación debe estar asegurada con protocolo HTTPS

\item {} 
\sphinxAtStartPar
La contraseña debe cumplir con un patrón de mínimo 8 caracteres que incluyan una mayúscula, una minúscula y un número

\item {} 
\sphinxAtStartPar
La contraseña debe ser almacenada encripatada con algún algoritmo estándar

\item {} 
\sphinxAtStartPar
Los correos emitidos deben estar certificados y con estampa cronológica

\end{itemize}


\subsection{Requerimientos de mantenimiento}
\label{\detokenize{SRS:requerimientos-de-mantenimiento}}\begin{itemize}
\item {} 
\sphinxAtStartPar
Se espera un crecimiento de 200\% en el almacenamiento de los documentos, archivos de carga

\end{itemize}

\sphinxAtStartPar
y la data
\sphinxhyphen{} Se debe realizar backup diario de la base de datos
\sphinxhyphen{} Se debe contemplar el cambio fácil de motor de base de datos
\sphinxhyphen{} Cada acción del CRUD debe registrar su respectivo movimiento de auditoría con la
siguiente información: 1) Acción, 2) Usuario, 3) Fecha, 4) Hora
\sphinxhyphen{} Cada error producido en el sistema debe quedar registrado en un log


\subsection{Requerimientos de portabilidad}
\label{\detokenize{SRS:requerimientos-de-portabilidad}}
\sphinxAtStartPar
No se solicitaron requerimientos de portabilidad.


\chapter{Requerimientos específicos}
\label{\detokenize{SRS:requerimientos-especificos}}

\begin{savenotes}\sphinxatlongtablestart\begin{longtable}[c]{|*{6}{\X{1}{6}|}}
\sphinxthelongtablecaptionisattop
\caption{Requerimientos específicos\strut}\label{\detokenize{SRS:id1}}\\*[\sphinxlongtablecapskipadjust]
\hline
\sphinxstyletheadfamily 
\sphinxAtStartPar
Cód
&\sphinxstyletheadfamily 
\sphinxAtStartPar
UC
&\sphinxstyletheadfamily 
\sphinxAtStartPar
Actores
&\sphinxstyletheadfamily 
\sphinxAtStartPar
Procesamiento
&\sphinxstyletheadfamily 
\sphinxAtStartPar
Salidas
&\sphinxstyletheadfamily 
\sphinxAtStartPar
Excepciones
\\
\hline
\endfirsthead

\multicolumn{6}{c}%
{\makebox[0pt]{\sphinxtablecontinued{\tablename\ \thetable{} \textendash{} proviene de la página anterior}}}\\
\hline
\sphinxstyletheadfamily 
\sphinxAtStartPar
Cód
&\sphinxstyletheadfamily 
\sphinxAtStartPar
UC
&\sphinxstyletheadfamily 
\sphinxAtStartPar
Actores
&\sphinxstyletheadfamily 
\sphinxAtStartPar
Procesamiento
&\sphinxstyletheadfamily 
\sphinxAtStartPar
Salidas
&\sphinxstyletheadfamily 
\sphinxAtStartPar
Excepciones
\\
\hline
\endhead

\hline
\multicolumn{6}{r}{\makebox[0pt][r]{\sphinxtablecontinued{continué en la próxima página}}}\\
\endfoot

\endlastfoot

\sphinxAtStartPar
UC\sphinxhyphen{}1
&
\sphinxAtStartPar
Registrar Platos
&
\sphinxAtStartPar
Jefe de cocina
&
\sphinxAtStartPar
El jefe de cocina de cada uno de los puntos del restaurante en la ciudad puede registrar los platos o menu de este para lo cual debe ingresar 1)Categoria, 2)Producto, 3)Cantidad, 4)Unidad 5)Precio
&
\sphinxAtStartPar
El plato se registra  en BD
&
\sphinxAtStartPar
La vista se recarga nuevamente, pero se muestran errores de validación por cada campo del formulario
\\
\hline
\sphinxAtStartPar
UC\sphinxhyphen{}2
&
\sphinxAtStartPar
Registrar Pedido
&
\sphinxAtStartPar
Mesero
&
\sphinxAtStartPar
Llena un formulario con 1) Plato, 2) Cantidad 3) Mesa
&
\sphinxAtStartPar
El pedido se registra en BD y se descuentan unidades de producto según el plato
&
\sphinxAtStartPar
La vista se recarga nuevamente, pero se muestran errores de validación por cada campo del formulario
\\
\hline
\sphinxAtStartPar
UC\sphinxhyphen{}3
&
\sphinxAtStartPar
Solicitar stock de alimentos
&
\sphinxAtStartPar
Jefe de cocina
&
\sphinxAtStartPar
Llena un formulario múltiple con los campos 1) Categoría de producto, 2) Producto (lista desplegable que depende de 1), 3) Cantidad, 4) Marca Unidad
&
\sphinxAtStartPar
Se crea una solicitud, se registra en BD y luego se incluye el caso de uso ‘enviar correo’ destinado al solicitante y cuyo cuerpo es un mensaje de confirmación
&
\sphinxAtStartPar
La vista se recarga nuevamente, pero se muestran errores de validación por cada campo del formulario
\\
\hline
\sphinxAtStartPar
UC\sphinxhyphen{}4
&
\sphinxAtStartPar
Enviar Correo
&
\sphinxAtStartPar
n/a
&
\sphinxAtStartPar
Este UC es incluido dentro de otros casos de uso.
&
\sphinxAtStartPar
Se envía un correo con remitente, destinatario y cuerpo establecido según UC que lo incluya. Luego se añade un certificado y una estampa cronológica al final del cuerpo del correo.
&
\sphinxAtStartPar
Se envía un correo al remitente indicando que el mensaje no fue enviado y mostrando el correspondiente mensaje de error
\\
\hline
\sphinxAtStartPar
UC\sphinxhyphen{}5
&
\sphinxAtStartPar
Consulta solicitudes de sucursal
&
\sphinxAtStartPar
Auxiliar de compras
&
\sphinxAtStartPar
Visualiza una lista de solicitudes de stock por parte de las diversas sucursales. La vista se puede navegar por fecha de solicitud, Producto, Cantidad, Marca, y estado.
&
\sphinxAtStartPar
Desde esta vista, se puede seleccionar una solicitud de stock particular y entrar al UC ‘Generar solicitud de cotización.
&
\sphinxAtStartPar
n/a
\\
\hline
\sphinxAtStartPar
UC\sphinxhyphen{}6
&
\sphinxAtStartPar
Generar solicitud de cotización
&
\sphinxAtStartPar
Auxiliar de compras
&
\sphinxAtStartPar
Se estipula un proveedor (lista desplegable) y un rango de fechas.
&
\sphinxAtStartPar
La solicitud de cotización cambia a estado ‘No Enviada’ y ahora se puede consultar
&\\
\hline
\sphinxAtStartPar
UC\sphinxhyphen{}7
&
\sphinxAtStartPar
Enviar solicitud de cotización
&
\sphinxAtStartPar
Auxiliar de compras
&
\sphinxAtStartPar
Se seleccióna una solicitud de cotización que no esté en estado ‘enviada’ y se presiona el boton ‘enviar al proveedor’
&
\sphinxAtStartPar
La solicitud de cotización cambia a estado ‘Enviada’ y se ejecutael UC ‘enviar correo’ destinado al proveedor y ‘Actualiza solicitud con precios estandarizados’
&\\
\hline
\sphinxAtStartPar
UC\sphinxhyphen{}8
&
\sphinxAtStartPar
Actualizar solicitud con valores estandarizadas
&
\sphinxAtStartPar
n/a
&
\sphinxAtStartPar
Este UC se ejecuta automáticamente al enviar una solicitud. El sistma consulta un servicio expuesto por la entidad gubernamental y añade campo ‘precio sugerido’ a cada producto de la cotización.
&
\sphinxAtStartPar
Se actualizan valores de la solicitud en BD, y se deja un registro en el log del sistema de la actualizacións satisfactoria
&
\sphinxAtStartPar
Si hay un problema con el servicio de actualización de datos. El estado de la solicitud se revierte y se muestra un mensaje
\\
\hline
\sphinxAtStartPar
UC\sphinxhyphen{}9
&
\sphinxAtStartPar
Consulta respuesta de proveedores
&
\sphinxAtStartPar
Director de Compras
&
\sphinxAtStartPar
Recibe en su correo la respuesta de los proveedores. Esta parte del UC es manual y depende enteramente del actor.
&
\sphinxAtStartPar
n/a
&
\sphinxAtStartPar
n/a
\\
\hline
\sphinxAtStartPar
UC\sphinxhyphen{}10
&
\sphinxAtStartPar
Registra valor del proveedor en cotización
&
\sphinxAtStartPar
Director de Compras
&
\sphinxAtStartPar
Luego abre el formulario de la solicitud de cotización respectiva y agrega los valores propuestos por los proveedores en una columna vacía.
&
\sphinxAtStartPar
n/a
&
\sphinxAtStartPar
n/a
\\
\hline
\sphinxAtStartPar
UC\sphinxhyphen{}11
&
\sphinxAtStartPar
Valida cotización de proveedores con generada
&
\sphinxAtStartPar
Director de Compras
&
\sphinxAtStartPar
Por último, presiona el botón validar, el cual ejecutará la lógica de negocio correspondiente.
&
\sphinxAtStartPar
Si el valor de la cotización está por encima en más de un 25\% del valor estaándar del ministerio, la cotización queda en estado “RECHAZADA”, si la cotización está por debajo en más de un 50\% del valor estándar del ministerio, la cotización queda en estado “SOSPECHOSA” y finalmente las que se encuentren dentro de este rango quedan en estado “OPCIONADA”.
&
\sphinxAtStartPar
Si faltó un valor por agregar por parte del actor, el formulario se recargará con los valores ingresados y señalará el valor faltante.
\\
\hline
\sphinxAtStartPar
UC\sphinxhyphen{}12
&
\sphinxAtStartPar
Consulta tablero de control
&
\sphinxAtStartPar
Director de Compras
&
\sphinxAtStartPar
Vista con tablero de control (BAM) donde muestre los principales KPI del negocio
&\begin{enumerate}
\sphinxsetlistlabels{\arabic}{enumi}{enumii}{}{.}%
\item {} 
\sphinxAtStartPar
Producto con mayor demanda en el último mes, 2. Restaurante con mayor demanda en el último mes, 3. Proveedor con mejores precios en el último año

\end{enumerate}
&
\sphinxAtStartPar
n/a
\\
\hline
\sphinxAtStartPar
UC\sphinxhyphen{}13
&
\sphinxAtStartPar
Parametrizar validaciones
&
\sphinxAtStartPar
Administrador
&
\sphinxAtStartPar
Vista permite alterar los márgenes para determinar si una solicitud es “SOSPECHOSA”, “OPCIONADA” o “RECHAZADA”
&
\sphinxAtStartPar
Los márgenes se actualizan en BD.
&
\sphinxAtStartPar
Si los márgenes son inconsistentes (el mismo valor para dos estados, o el margen para aceptar es más grande que el de rechazar) la operación no tiene efecto y se muestra un mensaje de error.
\\
\hline
\sphinxAtStartPar
UC\sphinxhyphen{}14
&
\sphinxAtStartPar
Monitorear auditoría
&
\sphinxAtStartPar
Administador
&
\sphinxAtStartPar
Vista permite visualizar todas las tablas de la base de datos, con todos sus registros.
&
\sphinxAtStartPar
Al ingresar en el detalle de cada fila, se muestra una lista de cambios a cada valor
&
\sphinxAtStartPar
n/a
\\
\hline
\sphinxAtStartPar
UC\sphinxhyphen{}15
&
\sphinxAtStartPar
Consultar logs
&
\sphinxAtStartPar
Administrador
&
\sphinxAtStartPar
Vista muestra cada arhico de log existente
&
\sphinxAtStartPar
Al entrar al dtalle de cada archivo de log, se puede descargar el archivo .txt
&
\sphinxAtStartPar
n/a
\\
\hline
\end{longtable}\sphinxatlongtableend\end{savenotes}


\chapter{Documento de Arquitectura}
\label{\detokenize{SAD:documento-de-arquitectura}}\label{\detokenize{SAD::doc}}

\section{Intoducción}
\label{\detokenize{SAD:intoduccion}}
\sphinxAtStartPar
El propósito de este documento es proveer una visión
exhaustiva del Sistema de Información de Konrad Gourmet,para
lo cual se utilizarán diferentes vistas arquitectónicas
. Pretende capturar y plasmar las decisiones que se han
hecho en el sistema.


\section{Representación arquitectónica}
\label{\detokenize{SAD:representacion-arquitectonica}}
\sphinxAtStartPar
Para describir el sistema, se han desarrollado
4 vistas:
\begin{enumerate}
\sphinxsetlistlabels{\arabic}{enumi}{enumii}{}{.}%
\item {} 
\sphinxAtStartPar
Vista lógica

\item {} 
\sphinxAtStartPar
Vista de procesos

\item {} 
\sphinxAtStartPar
Vista de despliegue

\item {} 
\sphinxAtStartPar
Vista de implementación

\end{enumerate}


\subsection{Vista lógica}
\label{\detokenize{SAD:vista-logica}}
\sphinxAtStartPar
El propósito de esta vista es mostrar, a nivel general,
los componentes de software utilizados para el
funcionamiento del sistema. Para resaltar, se
utilizara el framework Django como backend, puesto
que tiene ya bien integrados las funcionalidades estipuladas
en los RNF como autenticación, cifrado, capacidad para
activar auditoría, etc. Además, en el front\sphinxhyphen{}end se utilizará
un framework de Javascript: Vue.js, por su ligereza y
facilida de integración con Django. Los componentes de
front\sphinxhyphen{}end son necesarios para cumplir los RNF de usabilidad.

\noindent\sphinxincludegraphics[width=1000\sphinxpxdimen]{{vista_logica}.png}

\sphinxAtStartPar
Esta es otra forma diferente de ver los componentes.
El orden de las capas debe ser entendido como el
orden de las llamadas de un componente a otro. Así
, la lógica del negocio “llama a” los eventos
y no viceversa, etc.

\noindent\sphinxincludegraphics[width=1000\sphinxpxdimen]{{vista_capas}.png}


\subsection{Vista de despliegue}
\label{\detokenize{SAD:vista-de-despliegue}}
\sphinxAtStartPar
Muestra los nodos significativos de la infraestructura
necesaria para el funcionamiento del sistema. Se decidió utilizar
un modelo IaS, puesto que sabemos la cantidad de sucursales
y usuarios en la central. No se pronostica mucho crecimiento, por
lo que la elasticidad de un PaS no se consideró necesaria.

\noindent\sphinxincludegraphics[width=1000\sphinxpxdimen]{{vista_despliegue}.png}


\subsection{Vista procesos}
\label{\detokenize{SAD:vista-procesos}}
\sphinxAtStartPar
Los siguientes diagramas de actividad muestran
a alto nivel el funcionamiento de algunos
casos de uso importantes dentro del sistema.
No pretenden ser exhaustivos, sino una guía
de como se activan funcionalidades automáticas
del sistema que están encapsulados dentro del
componente “eventos”.

\begin{figure}[htbp]
\centering
\capstart

\noindent\sphinxincludegraphics[width=600\sphinxpxdimen]{{vista_procesos}.png}
\caption{UC “Registrar pedido”}\label{\detokenize{SAD:id1}}
\begin{sphinxlegend}\begin{quote}
\end{quote}
\end{sphinxlegend}
\end{figure}

\begin{figure}[htbp]
\centering
\capstart

\noindent\sphinxincludegraphics[width=600\sphinxpxdimen]{{vista_procesos_2}.png}
\caption{UC “Enviar solicitud de cotización”}\label{\detokenize{SAD:id2}}
\begin{sphinxlegend}\begin{quote}
\end{quote}
\end{sphinxlegend}
\end{figure}

\begin{figure}[htbp]
\centering
\capstart

\noindent\sphinxincludegraphics[width=600\sphinxpxdimen]{{vista_procesos_3}.png}
\caption{UC “Validar cotización”}\label{\detokenize{SAD:id3}}
\begin{sphinxlegend}\begin{quote}
\end{quote}
\end{sphinxlegend}
\end{figure}


\subsection{Vista implementación}
\label{\detokenize{SAD:vista-implementacion}}
\sphinxAtStartPar
Esta vista provee una visión cercana al programador.
Los componentes del sistema se expresen en cuanto
a sus clases e interfaces principales.

\begin{figure}[htbp]
\centering

\noindent\sphinxincludegraphics[width=600\sphinxpxdimen]{{vista_implementacion}.png}
\end{figure}

\begin{figure}[htbp]
\centering

\noindent\sphinxincludegraphics[width=600\sphinxpxdimen]{{vista_implementacion_001}.png}
\end{figure}

\begin{figure}[htbp]
\centering

\noindent\sphinxincludegraphics[width=600\sphinxpxdimen]{{vista_implementacion_002}.png}
\end{figure}

\begin{figure}[htbp]
\centering

\noindent\sphinxincludegraphics[width=600\sphinxpxdimen]{{vista_implementacion_003}.png}
\end{figure}


\chapter{Modelo De datos}
\label{\detokenize{BD:modelo-de-datos}}\label{\detokenize{BD::doc}}
\noindent\sphinxincludegraphics[width=1000\sphinxpxdimen]{{base_datos}.png}


\chapter{Plan de pruebas}
\label{\detokenize{STD:plan-de-pruebas}}\label{\detokenize{STD::doc}}

\chapter{Casos de prueba}
\label{\detokenize{STD:casos-de-prueba}}

\begin{savenotes}\sphinxatlongtablestart\begin{longtable}[c]{|*{8}{\X{1}{8}|}}
\sphinxthelongtablecaptionisattop
\caption{Casos de prueba\strut}\label{\detokenize{STD:id1}}\\*[\sphinxlongtablecapskipadjust]
\hline
\sphinxstyletheadfamily 
\sphinxAtStartPar
Id
&\sphinxstyletheadfamily 
\sphinxAtStartPar
Caso de Prueba
&\sphinxstyletheadfamily 
\sphinxAtStartPar
Descripción
&\sphinxstyletheadfamily 
\sphinxAtStartPar
Funcionalidad / Característica
&\sphinxstyletheadfamily 
\sphinxAtStartPar
Datos / Acciones de Entrada
&\sphinxstyletheadfamily 
\sphinxAtStartPar
Resultado Esperado
&\sphinxstyletheadfamily 
\sphinxAtStartPar
Requerimientos de Ambiente de Pruebas
&\sphinxstyletheadfamily 
\sphinxAtStartPar
Procedimientos especiales requeridos
\\
\hline
\endfirsthead

\multicolumn{8}{c}%
{\makebox[0pt]{\sphinxtablecontinued{\tablename\ \thetable{} \textendash{} proviene de la página anterior}}}\\
\hline
\sphinxstyletheadfamily 
\sphinxAtStartPar
Id
&\sphinxstyletheadfamily 
\sphinxAtStartPar
Caso de Prueba
&\sphinxstyletheadfamily 
\sphinxAtStartPar
Descripción
&\sphinxstyletheadfamily 
\sphinxAtStartPar
Funcionalidad / Característica
&\sphinxstyletheadfamily 
\sphinxAtStartPar
Datos / Acciones de Entrada
&\sphinxstyletheadfamily 
\sphinxAtStartPar
Resultado Esperado
&\sphinxstyletheadfamily 
\sphinxAtStartPar
Requerimientos de Ambiente de Pruebas
&\sphinxstyletheadfamily 
\sphinxAtStartPar
Procedimientos especiales requeridos
\\
\hline
\endhead

\hline
\multicolumn{8}{r}{\makebox[0pt][r]{\sphinxtablecontinued{continué en la próxima página}}}\\
\endfoot

\endlastfoot

\sphinxAtStartPar
1
&
\sphinxAtStartPar
Registro exitoso de platos
&
\sphinxAtStartPar
Cuando el jefe de cocina ingresa al formulario y diligencia:
\begin{enumerate}
\sphinxsetlistlabels{\arabic}{enumi}{enumii}{}{.}%
\item {} 
\sphinxAtStartPar
Categoría

\item {} 
\sphinxAtStartPar
Producto

\item {} 
\sphinxAtStartPar
Cantidad

\item {} 
\sphinxAtStartPar
Unidad

\item {} 
\sphinxAtStartPar
Precio

\end{enumerate}

\sphinxAtStartPar
El plato debe registrarse correctamente en la base de datos
&
\sphinxAtStartPar
CU: Registrar platos
&\begin{enumerate}
\sphinxsetlistlabels{\arabic}{enumi}{enumii}{}{.}%
\item {} 
\sphinxAtStartPar
Categoría

\item {} 
\sphinxAtStartPar
Producto

\item {} 
\sphinxAtStartPar
Cantidad

\item {} 
\sphinxAtStartPar
Unidad

\item {} 
\sphinxAtStartPar
Precio

\end{enumerate}
&
\sphinxAtStartPar
El plato se registra en la base de datos y puede ser consultado.
&
\sphinxAtStartPar
N/A
&
\sphinxAtStartPar
N/A
\\
\hline
\sphinxAtStartPar
2
&
\sphinxAtStartPar
Registro exitoso de pedidos
&
\sphinxAtStartPar
Cuando el mesero ingresa al formulario y diligencia:
\begin{enumerate}
\sphinxsetlistlabels{\arabic}{enumi}{enumii}{}{.}%
\item {} 
\sphinxAtStartPar
Plato

\item {} 
\sphinxAtStartPar
Cantidad

\item {} 
\sphinxAtStartPar
Mesa

\end{enumerate}

\sphinxAtStartPar
El pedido debe registrarse correctamente en la base de datos, seguido, la base de datos descuenta las unidades de producto correspondientes al plato
&
\sphinxAtStartPar
CU: Registrar pedido
&\begin{enumerate}
\sphinxsetlistlabels{\arabic}{enumi}{enumii}{}{.}%
\item {} 
\sphinxAtStartPar
Plato

\item {} 
\sphinxAtStartPar
Cantidad

\item {} 
\sphinxAtStartPar
Mesa

\end{enumerate}
&
\sphinxAtStartPar
El pedido se registra en la base de datos y esta a su vez descuenta las unidades de los productos asociados al plato.
&
\sphinxAtStartPar
N/A
&
\sphinxAtStartPar
N/A
\\
\hline
\sphinxAtStartPar
3
&
\sphinxAtStartPar
Solicitud exitosa de stock de alimentos
&
\sphinxAtStartPar
Cuando el jefe de cocina ingresa al formulario y diligencia:
\begin{enumerate}
\sphinxsetlistlabels{\arabic}{enumi}{enumii}{}{.}%
\item {} 
\sphinxAtStartPar
Categoría de producto

\item {} 
\sphinxAtStartPar
Producto (según 1)

\item {} 
\sphinxAtStartPar
Cantidad

\item {} 
\sphinxAtStartPar
Marca

\item {} 
\sphinxAtStartPar
Unidad

\end{enumerate}

\sphinxAtStartPar
La solicitud debe registrarse en la base de datos y se envía un correo electrónico, indicando remitente, destinatario y con cuerpo: «Solicitar stock de alimentos». También se adiciona al cuerpo del correo un certificado y una estampa cronológica
&
\sphinxAtStartPar
CU: Solicitar stock de alimentos
CU: Enviar correo
&\begin{enumerate}
\sphinxsetlistlabels{\arabic}{enumi}{enumii}{}{.}%
\item {} 
\sphinxAtStartPar
Categoría de producto

\item {} 
\sphinxAtStartPar
Producto (según 1)

\item {} 
\sphinxAtStartPar
Cantidad

\item {} 
\sphinxAtStartPar
Marca

\item {} 
\sphinxAtStartPar
Unidad

\end{enumerate}
&\begin{enumerate}
\sphinxsetlistlabels{\arabic}{enumi}{enumii}{}{.}%
\item {} 
\sphinxAtStartPar
La solicitud se registra en la base de datos

\item {} 
\sphinxAtStartPar
El correo electrónico se envía a remitente con certificado y estampa cronológica

\end{enumerate}
&
\sphinxAtStartPar
N/A
&
\sphinxAtStartPar
N/A
\\
\hline
\sphinxAtStartPar
4
&
\sphinxAtStartPar
Consulta exitosa de solicitudes de sucursal
&
\sphinxAtStartPar
Cuando el auxiliar de compras navega por la vista de solicitudes de stock puede navegar por:
\begin{enumerate}
\sphinxsetlistlabels{\arabic}{enumi}{enumii}{}{.}%
\item {} 
\sphinxAtStartPar
Fecha de solicitud

\item {} 
\sphinxAtStartPar
Producto

\item {} 
\sphinxAtStartPar
Cantidad

\item {} 
\sphinxAtStartPar
Marca

\item {} 
\sphinxAtStartPar
Estado

\end{enumerate}

\sphinxAtStartPar
El sistema le permite seleccionar una solicitud de stock particular
&
\sphinxAtStartPar
CU: Consultar solicitudes de sucursal
&\begin{enumerate}
\sphinxsetlistlabels{\arabic}{enumi}{enumii}{}{.}%
\item {} 
\sphinxAtStartPar
Fecha de solicitud

\item {} 
\sphinxAtStartPar
Producto

\item {} 
\sphinxAtStartPar
Cantidad

\item {} 
\sphinxAtStartPar
Marca

\item {} 
\sphinxAtStartPar
Estado

\end{enumerate}
&
\sphinxAtStartPar
El sistema muestra las diferentes solicitudes
&
\sphinxAtStartPar
N/A
&
\sphinxAtStartPar
N/A
\\
\hline
\sphinxAtStartPar
5
&
\sphinxAtStartPar
Generación exitosa de solicitud de cotización
&
\sphinxAtStartPar
De acuerdo a la navegación y selección del caso de prueba 4, el auxiliar de compras diligencia:
\begin{enumerate}
\sphinxsetlistlabels{\arabic}{enumi}{enumii}{}{.}%
\item {} 
\sphinxAtStartPar
Proveedor

\item {} 
\sphinxAtStartPar
Fecha desde

\item {} 
\sphinxAtStartPar
Fecha hasta

\end{enumerate}
&
\sphinxAtStartPar
CU: Generar solicitud de cotización
&\begin{enumerate}
\sphinxsetlistlabels{\arabic}{enumi}{enumii}{}{.}%
\item {} 
\sphinxAtStartPar
Proveedor

\item {} 
\sphinxAtStartPar
Fecha desde

\item {} 
\sphinxAtStartPar
Fecha hasta

\end{enumerate}
&
\sphinxAtStartPar
El sistema cambia el estado de la cotización a «No Enviada»
&
\sphinxAtStartPar
N/A
&
\sphinxAtStartPar
N/A
\\
\hline
\sphinxAtStartPar
6
&
\sphinxAtStartPar
Envío exitoso de solicitud de cotización
&
\sphinxAtStartPar
Cuando el auxiliar de compras:
\begin{enumerate}
\sphinxsetlistlabels{\arabic}{enumi}{enumii}{}{.}%
\item {} 
\sphinxAtStartPar
Selecciona una solicitud de cotización en estado diferente a «Enviada»

\item {} 
\sphinxAtStartPar
Presiona el botón «Enviar al proveedor»

\end{enumerate}

\sphinxAtStartPar
El estado de la solicitud de cotización debe cambiar a «Enviada» y se envía un correo electrónico, indicando remitente, destinatario y con cuerpo: «Solicitar stock de alimentos». También se adiciona al cuerpo del correo un certificado y una estampa cronológica. Adicionalmente, el sistema consulta un servicio expuesto por la entidad gubernamental y añade un campo «Precio Sugerido» a cada producto incluido dentro de la cotización
&
\sphinxAtStartPar
CU: Enviar solicitud de cotización
CU: Enviar correo
CU: Actualizar solicitud con valores estandarizados
&\begin{enumerate}
\sphinxsetlistlabels{\arabic}{enumi}{enumii}{}{.}%
\item {} 
\sphinxAtStartPar
Solicitudes de cotizaciones de la base de datos

\item {} 
\sphinxAtStartPar
Botón «Enviar al proveedor»

\end{enumerate}
&\begin{enumerate}
\sphinxsetlistlabels{\arabic}{enumi}{enumii}{}{.}%
\item {} 
\sphinxAtStartPar
El estado de la cotización se actualiza en la base de datos

\item {} 
\sphinxAtStartPar
El correo electrónico se envía a remitente con certificado y estampa cronológica

\item {} 
\sphinxAtStartPar
El campo «precio sugerido» es añadido a cada producto incluido dentro de la cotización

\end{enumerate}
&
\sphinxAtStartPar
N/A
&
\sphinxAtStartPar
N/A
\\
\hline
\sphinxAtStartPar
7
&
\sphinxAtStartPar
Consulta exitosa de respuesta de proveedores
&
\sphinxAtStartPar
El director de compras recibe en su correo la respuesta de los proveedores
&
\sphinxAtStartPar
CU: Consulta respuesta de proveedores
&
\sphinxAtStartPar
Correo electrónico
&
\sphinxAtStartPar
N/A
&
\sphinxAtStartPar
N/A
&
\sphinxAtStartPar
N/A
\\
\hline
\sphinxAtStartPar
8
&
\sphinxAtStartPar
Registro exitoso de valores de proveedores en cotización
&
\sphinxAtStartPar
Para este caso, director de compras:
\begin{enumerate}
\sphinxsetlistlabels{\arabic}{enumi}{enumii}{}{.}%
\item {} 
\sphinxAtStartPar
Consulta la cotización respectiva

\item {} 
\sphinxAtStartPar
Agrega los valores propuestos por los proveedores en una columna vacía

\end{enumerate}
&
\sphinxAtStartPar
CU: Registrar valor del proveedor en cotización
&\begin{enumerate}
\sphinxsetlistlabels{\arabic}{enumi}{enumii}{}{.}%
\item {} 
\sphinxAtStartPar
Solicitudes de cotizaciones de la base de datos

\item {} 
\sphinxAtStartPar
Columna vacía para agregar valor

\end{enumerate}
&\begin{enumerate}
\sphinxsetlistlabels{\arabic}{enumi}{enumii}{}{.}%
\item {} 
\sphinxAtStartPar
El sistema trae la cotización indicada

\item {} 
\sphinxAtStartPar
El sistema permite la inclusión del valor propuesto dentro de la cotización

\end{enumerate}
&
\sphinxAtStartPar
N/A
&
\sphinxAtStartPar
N/A
\\
\hline
\sphinxAtStartPar
9
&
\sphinxAtStartPar
Validación exitosa de cotizaciones
&
\sphinxAtStartPar
Para este caso, director de compras:
\begin{enumerate}
\sphinxsetlistlabels{\arabic}{enumi}{enumii}{}{.}%
\item {} 
\sphinxAtStartPar
Ejecuta el botón «Validar»

\end{enumerate}
&
\sphinxAtStartPar
CU: Valida cotización de proveedores contra generada
&\begin{enumerate}
\sphinxsetlistlabels{\arabic}{enumi}{enumii}{}{.}%
\item {} 
\sphinxAtStartPar
Botón «Validar»

\end{enumerate}
&
\sphinxAtStartPar
La ejecución del botón permite la validación de la lógica de negocio explicada a continuación:
\begin{enumerate}
\sphinxsetlistlabels{\arabic}{enumi}{enumii}{}{.}%
\item {} 
\sphinxAtStartPar
Si el valor está por encima en más de un 25\% del valor estándar del Ministerio, el estado de la cotización quedará en estado «Rechazada»

\item {} 
\sphinxAtStartPar
Si el valor está por debajo en más de un 50\% del valor estándar del Ministerio, el estado de la cotización quedará en estado «Sospechosa»

\item {} 
\sphinxAtStartPar
Si el valor se encuentra dentro del rango, el estado de la cotización quedará en estado «Opcionada»

\end{enumerate}
&
\sphinxAtStartPar
N/A
&
\sphinxAtStartPar
N/A
\\
\hline
\sphinxAtStartPar
10
&
\sphinxAtStartPar
Consulta exitosa de tablero de control
&
\sphinxAtStartPar
El director de compras puede navegar por el tablero de control, y en este observar y filtrar:
\begin{enumerate}
\sphinxsetlistlabels{\arabic}{enumi}{enumii}{}{.}%
\item {} 
\sphinxAtStartPar
Producto con mayor demanda (Mes)

\item {} 
\sphinxAtStartPar
Restaurante con mayor demanda (Mes)

\item {} 
\sphinxAtStartPar
Proveedor con mejores precios (Año)

\end{enumerate}
&
\sphinxAtStartPar
CU: Consulta tablero de control
&
\sphinxAtStartPar
Módulo: Tablero de control
&
\sphinxAtStartPar
El sistema habilita una vista donde se muestran los principales KPI del negocio
&
\sphinxAtStartPar
N/A
&
\sphinxAtStartPar
N/A
\\
\hline
\sphinxAtStartPar
11
&
\sphinxAtStartPar
Parametrización exitosa de validaciones
&
\sphinxAtStartPar
El administrador puede navegar por el módulo y alterar los márgenes de estados de solicitudes de cotización
&
\sphinxAtStartPar
CU: Parametrizar validaciones
&
\sphinxAtStartPar
Módulo: Parametrización
&
\sphinxAtStartPar
El sistema habilita una vista donde el administrador puede actualizar los valores de las márgenes y estos quedan correctamente actualizados en la base de datos
&
\sphinxAtStartPar
N/A
&
\sphinxAtStartPar
N/A
\\
\hline
\sphinxAtStartPar
12
&
\sphinxAtStartPar
Monitoreo exitoso de audioría
&
\sphinxAtStartPar
El administrador puede navegar por el módulo y visualizar todas las tablas de la base de datos
&
\sphinxAtStartPar
CU: Monitorear auditoría
&
\sphinxAtStartPar
Módulo: Auditoría
&
\sphinxAtStartPar
El sistema habilita una vista donde el administrador puede visualizar todas las tablas de la base de datos, con todos sus registros
&
\sphinxAtStartPar
N/A
&
\sphinxAtStartPar
N/A
\\
\hline
\sphinxAtStartPar
13
&
\sphinxAtStartPar
Consulta exitosa de archivos de log
&
\sphinxAtStartPar
El administrador puede navegar por el módulo y visualizar todos los archivos de log existentes
&
\sphinxAtStartPar
CU: Consultar logs
&
\sphinxAtStartPar
Módulo: Auditoría
&
\sphinxAtStartPar
El sistema habilita una vista donde el administrador puede visualizar todos los archivos existentes de logs
&
\sphinxAtStartPar
N/A
&
\sphinxAtStartPar
N/A
\\
\hline
\end{longtable}\sphinxatlongtableend\end{savenotes}


\chapter{Prototipos}
\label{\detokenize{prototipos:prototipos}}\label{\detokenize{prototipos::doc}}

\section{Web}
\label{\detokenize{prototipos:web}}
\noindent\sphinxincludegraphics[width=1000\sphinxpxdimen]{{1}.jpg}

\noindent\sphinxincludegraphics[width=1000\sphinxpxdimen]{{2}.jpg}

\noindent\sphinxincludegraphics[width=1000\sphinxpxdimen]{{3}.jpg}

\noindent\sphinxincludegraphics[width=1000\sphinxpxdimen]{{4}.jpg}

\noindent\sphinxincludegraphics[width=1000\sphinxpxdimen]{{5}.jpg}

\noindent\sphinxincludegraphics[width=1000\sphinxpxdimen]{{6}.jpg}

\noindent\sphinxincludegraphics[width=1000\sphinxpxdimen]{{7}.jpg}

\noindent\sphinxincludegraphics[width=1000\sphinxpxdimen]{{8}.jpg}

\noindent\sphinxincludegraphics[width=1000\sphinxpxdimen]{{9}.jpg}

\noindent\sphinxincludegraphics[width=1000\sphinxpxdimen]{{10}.jpg}

\noindent\sphinxincludegraphics[width=1000\sphinxpxdimen]{{11}.jpg}


\section{Móvil}
\label{\detokenize{prototipos:movil}}
\noindent\sphinxincludegraphics[width=1000\sphinxpxdimen]{{12}.jpg}

\noindent\sphinxincludegraphics[width=1000\sphinxpxdimen]{{21}.jpg}

\noindent\sphinxincludegraphics[width=1000\sphinxpxdimen]{{31}.jpg}

\noindent\sphinxincludegraphics[width=1000\sphinxpxdimen]{{31}.jpg}


\chapter{Indices and tables}
\label{\detokenize{index:indices-and-tables}}\begin{itemize}
\item {} 
\sphinxAtStartPar
\DUrole{xref,std,std-ref}{genindex}

\item {} 
\sphinxAtStartPar
\DUrole{xref,std,std-ref}{modindex}

\item {} 
\sphinxAtStartPar
\DUrole{xref,std,std-ref}{search}

\end{itemize}



\renewcommand{\indexname}{Índice}
\printindex
\end{document}