%% Generated by Sphinx.
\def\sphinxdocclass{report}
\documentclass[letterpaper,10pt,spanish]{sphinxmanual}
\ifdefined\pdfpxdimen
   \let\sphinxpxdimen\pdfpxdimen\else\newdimen\sphinxpxdimen
\fi \sphinxpxdimen=.75bp\relax
\ifdefined\pdfimageresolution
    \pdfimageresolution= \numexpr \dimexpr1in\relax/\sphinxpxdimen\relax
\fi
%% let collapsible pdf bookmarks panel have high depth per default
\PassOptionsToPackage{bookmarksdepth=5}{hyperref}

\PassOptionsToPackage{warn}{textcomp}
\usepackage[utf8]{inputenc}
\ifdefined\DeclareUnicodeCharacter
% support both utf8 and utf8x syntaxes
  \ifdefined\DeclareUnicodeCharacterAsOptional
    \def\sphinxDUC#1{\DeclareUnicodeCharacter{"#1}}
  \else
    \let\sphinxDUC\DeclareUnicodeCharacter
  \fi
  \sphinxDUC{00A0}{\nobreakspace}
  \sphinxDUC{2500}{\sphinxunichar{2500}}
  \sphinxDUC{2502}{\sphinxunichar{2502}}
  \sphinxDUC{2514}{\sphinxunichar{2514}}
  \sphinxDUC{251C}{\sphinxunichar{251C}}
  \sphinxDUC{2572}{\textbackslash}
\fi
\usepackage{cmap}
\usepackage[T1]{fontenc}
\usepackage{amsmath,amssymb,amstext}
\usepackage{babel}



\usepackage{tgtermes}
\usepackage{tgheros}
\renewcommand{\ttdefault}{txtt}



\usepackage[Sonny]{fncychap}
\ChNameVar{\Large\normalfont\sffamily}
\ChTitleVar{\Large\normalfont\sffamily}
\usepackage{sphinx}

\fvset{fontsize=auto}
\usepackage{geometry}


% Include hyperref last.
\usepackage{hyperref}
% Fix anchor placement for figures with captions.
\usepackage{hypcap}% it must be loaded after hyperref.
% Set up styles of URL: it should be placed after hyperref.
\urlstyle{same}

\addto\captionsspanish{\renewcommand{\contentsname}{Contents:}}

\usepackage{sphinxmessages}
\setcounter{tocdepth}{1}



\title{Konrad Gourmet}
\date{25 de octubre de 2021}
\release{0.0}
\author{Jonatan Ahumada}
\newcommand{\sphinxlogo}{\vbox{}}
\renewcommand{\releasename}{Versión}
\makeindex
\begin{document}

\ifdefined\shorthandoff
  \ifnum\catcode`\=\string=\active\shorthandoff{=}\fi
  \ifnum\catcode`\"=\active\shorthandoff{"}\fi
\fi

\pagestyle{empty}
\sphinxmaketitle
\pagestyle{plain}
\sphinxtableofcontents
\pagestyle{normal}
\phantomsection\label{\detokenize{index::doc}}



\chapter{SRS}
\label{\detokenize{SRS:srs}}\label{\detokenize{SRS::doc}}

\section{Introducción}
\label{\detokenize{SRS:introduccion}}
\sphinxAtStartPar
El propósito de este SRS es
describir los requerimientos
funcionales y no funcionales
del sistema de abastecimiento
de alimentos para la cadena
Konrad Gourmet. Este documento
está dirigido a los miembros
del equipo de desarrollo
y a los interesados. Los
documentos especificados
aquí serán incluidos en el release
1.0


\section{Ámbito del proyecto}
\label{\detokenize{SRS:ambito-del-proyecto}}
\sphinxAtStartPar
Konrad Gorumet es una cadena de
restaurantes que cuenta con varias sucurales.
Anteriormente, el proceso para consultar y
actualizar su abastecimiento era manual,
por lo que era imposible consultar
cifras exactas en tiempo real de todo
su inventario (los alimentos necesarios para
los platos que ofrecen a su cliente final). Esto ocasionaba pérdidas
por contar con cifras extactas con las
que solicitar cotizaciones a sus proveedores.

\sphinxAtStartPar
A raiz de eso, surgió la necesidad de implementar
un sistema que permita seguir el inventario
de las sucursales de Konrad Gorumet y que también automatize
y optimize las negociociones con sus proveedores.


\section{Descripción general}
\label{\detokenize{SRS:descripcion-general}}
\sphinxAtStartPar
Konrad Gourmet es un sistema para
cuantificar el inventario de
varias sucursales de comida.
El sistema genera órdenes de compras,
facturas, y resúmenes del inventario
de distintas sucursales de Konrad
Gourmet. Una vez generada una cotización,
el sistema le provee al Director de Compras
facilidades para validar si la respuesta del
proveedor es factible o no, de acuerdo
a la respuesta de otros proveedores, así
como a los precios de los alimentos
publicados por la entidad gubernamental
correspondiente.


\section{Funcionalidades del sistema}
\label{\detokenize{SRS:funcionalidades-del-sistema}}
\sphinxAtStartPar
Los siguientes casos de uso
fueron identificados. Se clasifican
se agruparon para facilitar
su comprensión en 4 áreas:
\begin{enumerate}
\sphinxsetlistlabels{\arabic}{enumi}{enumii}{}{.}%
\item {} 
\sphinxAtStartPar
Sucursal

\item {} 
\sphinxAtStartPar
Central

\item {} 
\sphinxAtStartPar
Opciones de configuración

\item {} 
\sphinxAtStartPar
Eventos automáticos o de soporte

\end{enumerate}


\subsection{Sucursal}
\label{\detokenize{SRS:sucursal}}
\noindent\sphinxincludegraphics[width=500\sphinxpxdimen]{{casos_de_uso_001}.png}


\subsection{Central}
\label{\detokenize{SRS:central}}
\noindent\sphinxincludegraphics[width=500\sphinxpxdimen]{{casos_de_uso_002}.png}

\noindent\sphinxincludegraphics[width=500\sphinxpxdimen]{{casos_de_uso_003}.png}


\subsection{Opciones de configuración}
\label{\detokenize{SRS:opciones-de-configuracion}}
\noindent\sphinxincludegraphics[width=500\sphinxpxdimen]{{casos_de_uso_004}.png}


\subsection{Eventos automáticos o procesos de soporte}
\label{\detokenize{SRS:eventos-automaticos-o-procesos-de-soporte}}
\noindent\sphinxincludegraphics[width=500\sphinxpxdimen]{{casos_de_uso}.png}


\section{Requerimientos de interfaces externas}
\label{\detokenize{SRS:requerimientos-de-interfaces-externas}}

\subsection{Interfaces de software}
\label{\detokenize{SRS:interfaces-de-software}}\begin{itemize}
\item {} 
\sphinxAtStartPar
La tecnología utilizada debe ser de libre licenciamiento

\item {} 
\sphinxAtStartPar
Los lenguajes, frameworks y librerías deben ser las últimas versiones estables, reconocidas, con\sphinxhyphen{} soporte y de buenas prácticas

\item {} 
\sphinxAtStartPar
Cualquier servicio que se tenga que exponer hacia un sistema externo, se debe hacer a través de SOAP

\end{itemize}


\section{Atributos de calidad}
\label{\detokenize{SRS:atributos-de-calidad}}

\subsection{Requerimientos de usabilidad}
\label{\detokenize{SRS:requerimientos-de-usabilidad}}\begin{itemize}
\item {} 
\sphinxAtStartPar
El sistema debe ser “responisive”, para ser utilizado desde dispositivos móviles

\item {} 
\sphinxAtStartPar
El sistema debe poder cambiar de imagen corporativa de manera parametrizada, sin necesidad de

\end{itemize}

\sphinxAtStartPar
recurrir a un diseñador gráfico


\subsection{Requerimientos de confiabilidad}
\label{\detokenize{SRS:requerimientos-de-confiabilidad}}\begin{itemize}
\item {} 
\sphinxAtStartPar
El sistema debe poder recuperarse de desastres, debe tener un centro de datos alterno

\end{itemize}


\subsection{Requerimientos de disponibilidad}
\label{\detokenize{SRS:requerimientos-de-disponibilidad}}\begin{itemize}
\item {} 
\sphinxAtStartPar
El sistema debe ofrecer un 99,7\% de alta disponibilidad

\end{itemize}


\subsection{Requerimientos de desempeño}
\label{\detokenize{SRS:requerimientos-de-desempeno}}\begin{itemize}
\item {} 
\sphinxAtStartPar
El sistema debe estar en capacidad de atender a 200.000 usuarios concurrentes

\item {} 
\sphinxAtStartPar
El sisteba debe estar en capacidad de  procesar 1000 TPS

\end{itemize}


\subsection{Requerimientos de seguridad}
\label{\detokenize{SRS:requerimientos-de-seguridad}}\begin{itemize}
\item {} 
\sphinxAtStartPar
El sistema debe contar con un módulo de autenticación y autorización

\item {} 
\sphinxAtStartPar
El sistema debe poder manejar usuarios, perfiles o roles y permisos

\item {} 
\sphinxAtStartPar
Cualquier comunicación debe estar asegurada con protocolo HTTPS

\item {} 
\sphinxAtStartPar
La contraseña debe cumplir con un patrón de mínimo 8 caracteres que incluyan una mayúscula, una minúscula y un número

\item {} 
\sphinxAtStartPar
La contraseña debe ser almacenada encripatada con algún algoritmo estándar

\item {} 
\sphinxAtStartPar
Los correos emitidos deben estar certificados y con estampa cronológica

\end{itemize}


\subsection{Requerimientos de mantenimiento}
\label{\detokenize{SRS:requerimientos-de-mantenimiento}}\begin{itemize}
\item {} 
\sphinxAtStartPar
Se espera un crecimiento de 200\% en el almacenamiento de los documentos, archivos de carga

\end{itemize}

\sphinxAtStartPar
y la data
\sphinxhyphen{} Se debe realizar backup diario de la base de datos
\sphinxhyphen{} Se debe contemplar el cambio fácil de motor de base de datos
\sphinxhyphen{} Cada acción del CRUD debe registrar su respectivo movimiento de auditoría con la
siguiente información: 1) Acción, 2) Usuario, 3) Fecha, 4) Hora
\sphinxhyphen{} Cada error producido en el sistema debe quedar registrado en un log


\subsection{Requerimientos de portabilidad}
\label{\detokenize{SRS:requerimientos-de-portabilidad}}
\sphinxAtStartPar
No se solicitaron requerimientos de portabilidad.


\chapter{Requerimientos específicos}
\label{\detokenize{SRS:requerimientos-especificos}}

\begin{savenotes}\sphinxatlongtablestart\begin{longtable}[c]{|\X{10}{80}|\X{10}{80}|\X{20}{80}|\X{20}{80}|\X{20}{80}|}
\sphinxthelongtablecaptionisattop
\caption{Requerimientos específicos\strut}\label{\detokenize{SRS:id1}}\\*[\sphinxlongtablecapskipadjust]
\hline
\sphinxstyletheadfamily 
\sphinxAtStartPar
UC
&\sphinxstyletheadfamily 
\sphinxAtStartPar
Actores
&\sphinxstyletheadfamily 
\sphinxAtStartPar
Procesamiento
&\sphinxstyletheadfamily 
\sphinxAtStartPar
Salidas
&\sphinxstyletheadfamily 
\sphinxAtStartPar
Excepciones
\\
\hline
\endfirsthead

\multicolumn{5}{c}%
{\makebox[0pt]{\sphinxtablecontinued{\tablename\ \thetable{} \textendash{} proviene de la página anterior}}}\\
\hline
\sphinxstyletheadfamily 
\sphinxAtStartPar
UC
&\sphinxstyletheadfamily 
\sphinxAtStartPar
Actores
&\sphinxstyletheadfamily 
\sphinxAtStartPar
Procesamiento
&\sphinxstyletheadfamily 
\sphinxAtStartPar
Salidas
&\sphinxstyletheadfamily 
\sphinxAtStartPar
Excepciones
\\
\hline
\endhead

\hline
\multicolumn{5}{r}{\makebox[0pt][r]{\sphinxtablecontinued{continué en la próxima página}}}\\
\endfoot

\endlastfoot

\sphinxAtStartPar
Registrar Platos
&
\sphinxAtStartPar
Jefe de cocina
&
\sphinxAtStartPar
El jefe de cocina de cada uno de los puntos del restaurante en la ciudad puede registrar los platos o menu de este para lo cual debe ingresar los siguientes datos
&
\sphinxAtStartPar
El plato se registra  en BD
&
\sphinxAtStartPar
La vista se recarga nuevamente, pero se muestran errores de validación por cada campo del formulario
\\
\hline
\sphinxAtStartPar
Registrar Pedido
&
\sphinxAtStartPar
Mesero
&
\sphinxAtStartPar
Llena un formulario con 1) Plato, 2) Cantidad 3) Mesa
&
\sphinxAtStartPar
El pedido se registra en BD y se descuentan unidades de producto según el plato
&
\sphinxAtStartPar
La vista se recarga nuevamente, pero se muestran errores de validación por cada campo del formulario
\\
\hline
\sphinxAtStartPar
Solicitar stock de alimentos
&
\sphinxAtStartPar
Jefe de cocina
&
\sphinxAtStartPar
Llena un formulario múltiple con los campos 1) Categoría de producto, 2) Producto (lista desplegable que depende de 1), 3) Cantidad, 4) Marca Unidad
&
\sphinxAtStartPar
Se crea una solicitud, se registra en BD y luego se incluye el caso de uso ‘enviar correo’ destinado al solicitante y cuyo cuerpo es un mensaje de confirmación
&
\sphinxAtStartPar
La vista se recarga nuevamente, pero se muestran errores de validación por cada campo del formulario
\\
\hline
\sphinxAtStartPar
Enviar Correo
&
\sphinxAtStartPar
n/a
&
\sphinxAtStartPar
Este UC es incluido dentro de otros casos de uso.
&
\sphinxAtStartPar
Se envía un correo con remitente, destinatario y cuerpo establecido según UC que lo incluya. Luego se añade un certificado y una estampa cronológica al final del cuerpo del correo.
&
\sphinxAtStartPar
Se envía un correo al remitente indicando que el mensaje no fue enviado y mostrando el correspondiente mensaje de error
\\
\hline
\sphinxAtStartPar
Consulta solicitudes de sucursal
&
\sphinxAtStartPar
Auxiliar de compras
&
\sphinxAtStartPar
Visualiza una lista de solicitudes de stock por parte de las diversas sucursales. La vista se puede navegar por fecha de solicitud, Producto, Cantidad, Marca, y estado.
&
\sphinxAtStartPar
Desde esta vista, se puede seleccionar una solicitud de stock particular y entrar al UC ‘Generar solicitud de cotización.
&
\sphinxAtStartPar
n/a
\\
\hline
\sphinxAtStartPar
Generar solicitud de cotización
&
\sphinxAtStartPar
Auxiliar de compras
&
\sphinxAtStartPar
Se estipula un proveedor (lista desplegable) y un rango de fechas.
&
\sphinxAtStartPar
La solicitud de cotización cambia a estado ‘No Enviada’ y ahora se puede consultar
&\\
\hline
\sphinxAtStartPar
Enviar solicitud de cotización
&
\sphinxAtStartPar
Auxiliar de compras
&
\sphinxAtStartPar
Se seleccióna una solicitud de cotización que no esté en estado ‘enviada’ y se presiona el boton ‘enviar al proveedor’
&
\sphinxAtStartPar
La solicitud de cotización cambia a estado ‘Enviada’ y se ejecutael UC ‘enviar correo’ destinado al proveedor y ‘Actualiza solicitud con precios estandarizados’
&\\
\hline
\sphinxAtStartPar
Actualizar solicitud con valores estandarizadas
&
\sphinxAtStartPar
n/a
&
\sphinxAtStartPar
Este UC se ejecuta automáticamente al enviar una solicitud. El sistma consulta un servicio expuesto por la entidad gubernamental y añade campo ‘precio sugerido’ a cada producto de la cotización.
&
\sphinxAtStartPar
Se actualizan valores de la solicitud en BD, y se deja un registro en el log del sistema de la actualizacións satisfactoria
&
\sphinxAtStartPar
Si hay un problema con el servicio de actualización de datos. El estado de la solicitud se revierte y se muestra un mensaje
\\
\hline
\sphinxAtStartPar
Consulta respuesta de proveedores
&
\sphinxAtStartPar
Director de Compras
&
\sphinxAtStartPar
Recibe en su correo la respuesta de los proveedores. Esta parte del UC es manual y depende enteramente del actor.
&
\sphinxAtStartPar
n/a
&
\sphinxAtStartPar
n/a
\\
\hline
\sphinxAtStartPar
Registra valor del proveedor en cotización
&
\sphinxAtStartPar
Director de Compras
&
\sphinxAtStartPar
Luego abre el formulario de la solicitud de cotización respectiva y agrega los valores propuestos por los proveedores en una columna vacía.
&
\sphinxAtStartPar
n/a
&
\sphinxAtStartPar
n/a
\\
\hline
\sphinxAtStartPar
Valida cotización de proveedores con generada
&
\sphinxAtStartPar
Director de Compras
&
\sphinxAtStartPar
Por último, presiona el botón validar, el cual ejecutará la lógica de negocio correspondiente.
&
\sphinxAtStartPar
Si el valor de la cotización está por encima en más de un 25\% del valor estándar del ministerio, la cotización queda en estado “RECHAZADA”, si la cotización está por debajo en más de un 50\% del valor estándar del ministerio, la cotización queda en estado “SOSPECHOSA” y finalmente las que se encuentren dentro de este rango quedan en estado “OPCIONADA”.
&
\sphinxAtStartPar
Si faltó un valor por agregar por parte del actor, el formulario se recargará con los valores ingresados y señalará el valor faltante.
\\
\hline
\sphinxAtStartPar
Consulta tablero de control
&
\sphinxAtStartPar
Director de Compras
&
\sphinxAtStartPar
Vista con tablero de control (BAM) donde muestre los principales KPI del negocio
&\begin{enumerate}
\sphinxsetlistlabels{\arabic}{enumi}{enumii}{}{.}%
\item {} 
\sphinxAtStartPar
Producto con mayor demanda en el último mes, 2. Restaurante con mayor demanda en el último mes, 3. Proveedor con mejores precios en el último año

\end{enumerate}
&
\sphinxAtStartPar
n/a
\\
\hline
\sphinxAtStartPar
Parametrizar validaciones
&
\sphinxAtStartPar
Administrador
&
\sphinxAtStartPar
Vista permite alterar los márgenes para determinar si una solicitud es “SOSPECHOSA”, “OPCIONADA” o “RECHAZADA”
&
\sphinxAtStartPar
Los márgenes se actualizan en BD.
&
\sphinxAtStartPar
Si los márgenes son inconsistentes (el mismo valor para dos estados, o el margen para aceptar es más grande que el de rechazar) la operación no tiene efecto y se muestra un mensaje de error.
\\
\hline
\sphinxAtStartPar
Monitorear auditoría
&
\sphinxAtStartPar
Administador
&
\sphinxAtStartPar
Vista permite visualizar todas las tablas de la base de datos, con todos sus registros.
&
\sphinxAtStartPar
Al ingresar en el detalle de cada fila, se muestra una lista de cambios a cada valor
&
\sphinxAtStartPar
n/a
\\
\hline
\sphinxAtStartPar
Consultar logs
&
\sphinxAtStartPar
Administrador
&
\sphinxAtStartPar
Vista muestra cada arhico de log existente
&
\sphinxAtStartPar
Al entrar al dtalle de cada archivo de log, se puede descargar el archivo .txt
&
\sphinxAtStartPar
n/a
\\
\hline
\end{longtable}\sphinxatlongtableend\end{savenotes}


\chapter{Documento de Arquitectura}
\label{\detokenize{SAD:documento-de-arquitectura}}\label{\detokenize{SAD::doc}}

\section{Intoducción}
\label{\detokenize{SAD:intoduccion}}
\sphinxAtStartPar
El propósito de este documento es proveer una visión
exhaustiva del Sistema de Información de Konrad Gourmet,para
lo cual se utilizarán diferentes vistas arquitectónicas
. Pretende capturar y plasmar las decisiones que se han
hecho en el sistema.


\section{Representación arquitectónica}
\label{\detokenize{SAD:representacion-arquitectonica}}
\sphinxAtStartPar
Para describir el sistema, se han desarrollado
4 vistas:
\begin{enumerate}
\sphinxsetlistlabels{\arabic}{enumi}{enumii}{}{.}%
\item {} 
\sphinxAtStartPar
Vista lógica
\begin{quote}

\sphinxAtStartPar
Describe la conformación del software en cuanto sus componentes
\end{quote}

\item {} 
\sphinxAtStartPar
Vista de procesos

\item {} 
\sphinxAtStartPar
Vista de despliegue

\item {} 
\sphinxAtStartPar
Vista de implementación

\end{enumerate}


\subsection{Vista lógica}
\label{\detokenize{SAD:vista-logica}}
\sphinxAtStartPar
El propósito de esta vista es mostrar, a nivel general,
los componentes de software utilizados para el
funcionamiento del sistema. Para resaltar, se
utilizara el framework Django como backend, puesto
que tiene ya bien integrados las funcionalidades estipuladas
en los RNF como autenticación, cifrado, capacidad para
activar auditoría, etc. Además, en el front\sphinxhyphen{}end se utilizará
un framework de Javascript: Vue.js, por su ligereza y
facilida de integración con Django. Los componentes de
front\sphinxhyphen{}end son necesarios para cumplir los RNF de usabilidad.

\noindent\sphinxincludegraphics[width=1000\sphinxpxdimen]{{vista_logica}.png}


\subsection{Vista de despliegue}
\label{\detokenize{SAD:vista-de-despliegue}}
\noindent\sphinxincludegraphics[width=1000\sphinxpxdimen]{{vista_despliegue}.png}


\chapter{Indices and tables}
\label{\detokenize{index:indices-and-tables}}\begin{itemize}
\item {} 
\sphinxAtStartPar
\DUrole{xref,std,std-ref}{genindex}

\item {} 
\sphinxAtStartPar
\DUrole{xref,std,std-ref}{modindex}

\item {} 
\sphinxAtStartPar
\DUrole{xref,std,std-ref}{search}

\end{itemize}



\renewcommand{\indexname}{Índice}
\printindex
\end{document}